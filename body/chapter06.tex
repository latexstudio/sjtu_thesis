%%==================================================
%% chapter03.tex for SJTU Master Thesis
%% Encoding: UTF-8
%%==================================================

\chapter{总结与展望}
\label{chap:summary}

\section{全文总结}
\label{summary_summary}

本文首先总结和分析了云计算环境下的对称可搜索加密技术,包括对称可搜索加密中的各个子问题(精确搜索、模糊搜索、范围搜索、布尔搜索以及动态搜索);然后详细地分析了对称可搜索加密环境下的模糊搜索加密技术的研究现状和主要的研究内容,以及对该方案中的不足进行了阐述。通过对对称搜索环境下已有方案的研究,我们总结出已有方案普遍存在的一个安全缺陷 --- 搜索时信息泄漏过大的问题,并针对该问题提出了一个能降低在查找过程中信息泄漏的方案---即抗信息泄漏的可搜索加密方案(基于史密斯正交化的原理),该方案证明了在通过增加存储开销时减少信息的泄漏。除此之外,我们挖掘出该语境下一个新的复杂对称可搜索加密技术问题 --- 同义词对称可搜索加密技术;在文中我们阐述了同义词搜索与模糊搜索的关联。随之通过对该问题的详细分析和设计,我们提出了一个确保低通讯开销、少信息泄漏兼高搜索性能的同义词搜索方案;在方案中,我们引入了同义词集合和同义词判断函数,并且详细地描述了方案的算法实现和证明了方案的安全性。此外,我们阐述我们方案通讯开销为$O(1)$,查找开销为$O(p)$ (p --- 表示单词同义词集合的大小)。

综合上述问题,在文中,我们提出了两个解决方案---抗信息泄漏的可搜索加密和同义词可搜索加密。在方案中,我们证明了它们的可行性和性能以及安全性;并描述上述方案都能很好地应用于实际中已提出的方案之中,进一步扩展了已有方案的功能、安全性和可扩展性;但是,我们的方案中都对服务器端的存储开销有所增加。
%这使得我们的方案在云计算环境下,不仅达到了原有的理论要求并且有着远大的现实意义,甚至这将使得原有方案更好完善,使理论向现实又迈进了伟大的一篇。



\section{未来展望}
\label{summary_fulture}

在大数据时代的背景下,云计算技术正以着飞速的速度发展和应用。但是安全的云计算可搜索加密方案仍处于理论的阶段,虽然有些安全的云计算系统已被开发,但是不足以面对现实的需求,主要由于这些系统难以在可应用和强安全性之间达到平衡 --- 可用性好不够安全,而足够安全则功能太过于单一而不被使用,并且这些系统功能也非常单一。为了开发出兼容诸多优势的方案,我们还有很大一段距离要走 --- 主要是还有很多难题需要被提出和进一步研究。从本文的研究方向上来看,主要可以从如下几个方向进行进一步研究:

\begin{enumerate}
  \item
  在已实现的同义词可搜索加密方案中,并不能实现同义词集合随着上下文环境的变化而变化 --- 即同一份文档集在不同环境下有不同的同义词集,或在不同时间段同义词随着环境来进行动态改变。最终,希望在不久,能构建一个同义词动态变化的同义词可搜索加密方案。

  \item
  通过我们的方案的详细描述,我们了解了同义词和模糊搜索之间的紧密联系。接下来我们希望将同义词搜索和模糊搜索结合起来,实现一个确保安全的相似可搜索加密方案 --- 同时兼容模糊和同义词搜索功能。

  \item
  由于现有的模糊搜索方案都是基于通配符的解决方案,这些方案有一个明显的缺陷 --- 高存储开销。接下来,我们希望能提出一个低存储的模糊搜索方案。

  \item
  我们希望将我们的方案扩展到动态的场景下 --- 同时支持文档的动态的修改和同义词搜索,使方案更具可实现性和灵活性。
\end{enumerate}


